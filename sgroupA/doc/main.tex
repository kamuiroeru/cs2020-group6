\documentclass[uplatex]{jsarticle}
\usepackage[top=20truemm, bottom=20truemm, left=20truemm, right=20truemm]{geometry}
\usepackage{url}
\usepackage{booktabs}
\usepackage{here}
\usepackage{amsmath, amssymb}
\usepackage[dvipdfmx]{graphicx}

\title{
    \vspace{-1.5cm}
    サイバーセキュリティ Assignment 3 \\
    Incident 1
}
\author{group 6}

\begin{document}
\maketitle

\subsubsection*{担当者}
\begin{table}[H]
    \begin{tabular}{|c|l|l|}
        \hline
        学生番号 & 氏名 & 所属研究室\\
        \hline\hline
        1911403 & 佐々木 皓大 & ユビキタスコンピューティングシステム研究室\\
        \hline
        2011067 & 奥村 嶺 & 情報基盤システム学研究室(Inet-Lab)\\
        \hline
        2011115 & 佐伯 雄飛 & \\
        \hline
    \end{tabular}
\end{table}

\section*{Step 1}
\subsection*{Q}
For each incident, provide a detailed summary of the incident report toward cybersecurity professionals, by fully mobilizing concepts that you learned through these lectures. Create a timeline of the incident, as well as a fishbone diagram, that illustrates root cause and other important events in order to help with the understanding.
\subsection*{A}

\section*{Step 2}
\subsection*{Q}
Identify technical factors that led to each incident. Based on the real timeline of the particular incident, identify two critical periods where technical intervention was necessary.
\subsection*{A}

\section*{Step 3}
\subsection*{Q}
Identify human factors that led to each incident. Based on the real situation of the particular incident, describe how best you would deliver risk messages, as an external security consultant, in the two critical periods that you identified in the step 2.
\subsection*{A}

\end{document}